\documentclass[12pt,a4paper]{article}
\usepackage[utf8]{inputenc}
\usepackage{amsmath}
\usepackage{amsfonts}
\usepackage{amssymb}
\author{Matthias Englert, Fabian Schilha, Andreas Rottach}
\title{Pflichtenheft}
\begin{document}
\maketitle
\newpage
\tableofcontents
\newpage
\section{Überblick}
\subsection{Einleitung}
Dieses Software-Projekt hat sich als Ziel gesetzt eine webbasierte, zentrale E-Learning Plattform für die Studenten der Universität Ulm bereitzustellen. Das System soll die Lerninhalte individuell für jeden Benutzer in geeigneter Form strukturieren. Des Weiteren soll jeder Anwender den Lernstoff erweitern und mit anderen darüber diskutieren können. Die Lerninhalte sollen in einer hierarchischen Struktur mit verschiedenen Detailebenen dargestellt werden, um unterschiedliche Einblicke in ein Themengebiet zu ermöglichen. Das Skript soll durch verschiedene digitale Inhalte wie Bilder, Texte oder Videos unterstützt werden. Dozenten können initiale Lehrinhalte bereitstellen, die sich im Laufe des Semesters verändern oder erweitern werden können. 
Jeder Student soll die Möglichkeit haben, mit andern über Probleme zu diskutieren und Lösungen zu finden.

\subsection{Motivation}


\subsection{Vision und Leitbild}
Der Professor hat die Möglichkeit eine Veranstaltung anzulegen, auf der er dann ein initiales Skript bereitstellen kann. Durch die, während des Semesters aufkommenden Diskussionen, ist er in der Lage, das Skript mit Hilfe der Studenten zu erweitern. Er erkennt sofort, wenn es an bestimmten Stellen Probleme gibt und kann die Lösung ins Skript übernehmen. \\
Die Studenten helfen sich gegenseitig, indem sie sich in Diskussionen einklinken und an der Gestaltung der Lehrinhalte teilnehmen. Um die Qualität der Diskussion zu beurteilen zu können, gibt es die Möglichkeit, einzelne Beiträge durch positive Bewertungen hervorzuheben. Außerdem existieren Moderatoren, die die Aufgabe haben, schlechte Beiträge zu entfernen und besonders gute Beiträge ins Skript einzuarbeiten. Diese Rolle könnte beispielsweise der Professor oder ein Übungsleiter übernehmen. Durch die Verschiedenen eingebetteten Medien, soll das Lernen erleichtert werden und trockene Texte sollen durch Bilder und Videos aufbereitet werden. Da die Lerninhalte in kleinen Karteikarten gespeichert werden, ist das System in der Lage, die Informationen auf unterschiedlichste Weise darzustellen. Die Informationen lassen sich als Bäume oder Netze hierarchisch darstellen. Wichtige Informationen werden automatisch hervorgehoben, wenn der Ersteller sie als wichtig markiert. Die Studenten können sich somit einen Überblick über den Stoff verschaffen, sind leicht in der Lage Wichtiges von Zusatzinformationen zu trennen und erkennen über Querverweise welche anderen Kapitel für das aktuelle Thema relevant sind. \\
Wenn ein Student beispielsweise ein Matheskript liest und über den Begriff der Differenzialgleichung stößt, dann wird zuerst eine kurze Definition zum Begriff gegeben. Falls dies nicht ausreichend ist, hat er die Wahl sich zwischen verschiedenen Quellen zu diesem Thema zu entscheiden. Beispielsweise könnte er auf ein YouTube-Video oder eine andere Website verlinkt werden. 

\subsection{Projektkontext}
Das Software-System wird im Rahmen des Softwaregrundprojekts Wintersemester 2014/2015 im Bereich Informatik entstehen. Dies kann eventuell in den bestehenden Lehrbetrieb der Universität Ulm eingebettet werden, so dass allen Studenten an der Universität die Möglichkeit zu diesem System angeboten werden kann.

\section{Anforderungsanalyse}
\subsection{Fachwissen (Glossar)}

\subsection{Systemkontext}
\subsubsection{Akteure und Anwendungsfälle}
In diesem Abschnitt wird behandelt, welche Akteure existieren und welche Anwendungsfälle auftreten werden. 
hier eine Liste der beteiligten Akteure.

\begin{itemize}
\item Dozent
\item Student
\item Moderator
\item Tutor
\item Administrator
\end{itemize}

Folgenden Anwendungsfälle existieren:
\begin{itemize}
\item Ein Nutzer meldet sich im System an.
\item Der Dozent erstellt ein initiales Skript.
\item 
\end{itemize}
\subsubsection{Szenarien}
\subsubsection{Systemaufgabe}



\end{document}
